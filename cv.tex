% vim: set textwidth=120:

% Example CV based on the 1.5-column-cv template. Main features:
% * uses the Roboto font family and IcoMoon icon set;
% * doesn't use colours, different font weights are used instead for styling;
% * because the CV fits on one page, header and footer is empty, since there isn't much useful info to put there;
% * includes a photo.
\documentclass[a4paper,10pt]{article}


% package imports
% ---------------

\usepackage[british]{babel} % for correct language and hyphenation and stuff
\usepackage{calc}           % for easier length calculations (infix notation)
\usepackage{enumitem}       % for configuring list environments
\usepackage{fancyhdr}       % for setting header and footer
\usepackage{fontspec}       % for fonts
\usepackage{geometry}       % for setting margins (\newgeometry)
\usepackage{graphicx}       % for pictures
\usepackage{microtype}      % for microtypography stuff
\usepackage{xcolor}         % for colours


% margin and column widths
% ------------------------

% margins
\newgeometry{left=10mm,right=10mm,top=10mm,bottom=10mm}

% width of the gap between left and right column
\newlength{\cvcolumngapwidth}
\setlength{\cvcolumngapwidth}{3.5mm}

% left column width
\newlength{\cvleftcolumnwidth}
\setlength{\cvleftcolumnwidth}{44.5mm}

% right column width
\newlength{\cvrightcolumnwidth}
\setlength{\cvrightcolumnwidth}{\textwidth-\cvleftcolumnwidth-\cvcolumngapwidth}

% set paragraph indentation to 0, because it screws up the whole layout otherwise
\setlength{\parindent}{0mm}


% style definitions
% -----------------
% style categories explanation:
% * \cvnameXXX is used for the name;
% * \cvsectionXXX is used for section names (left column, accompanied by a horizontal rule);
% * \cvtitleXXX is used for job/education titles (right column);
% * \cvdurationXXX is used for job/education durations (left column);
% * \cvheadingXXX is used for headings (left column);
% * \cvmainXXX (and \setmainfont) is used for main text;
% * \cvruleXXX is used for the horizontal rules denoting sections.

% font families
\defaultfontfeatures{Ligatures=TeX} % reportedly a good idea, see https://tex.stackexchange.com/a/37251

\newfontfamily{\cvnamefont}{Roboto Medium}
\newfontfamily{\cvsectionfont}{Roboto Medium}
\newfontfamily{\cvtitlefont}{Roboto Regular}
\newfontfamily{\cvdurationfont}{Roboto Light Italic}
\newfontfamily{\cvheadingfont}{Roboto Regular}
\setmainfont{Roboto Light}

% colours
\definecolor{cvnamecolor}{HTML}{87001d}
\definecolor{cvsectioncolor}{HTML}{87001d}
\definecolor{cvtitlecolor}{HTML}{000000}
\definecolor{cvdurationcolor}{HTML}{000000}
\definecolor{cvheadingcolor}{HTML}{000000}
\definecolor{cvmaincolor}{HTML}{000000}
\definecolor{cvrulecolor}{HTML}{000000}

\color{cvmaincolor}

% styles
\newcommand{\cvnamestyle}[1]{{\Large\cvnamefont\textcolor{cvnamecolor}{#1}}}
\newcommand{\cvsectionstyle}[1]{{\normalsize\cvsectionfont\textcolor{cvsectioncolor}{#1}}}
\newcommand{\cvtitlestyle}[1]{{\large\cvtitlefont\textcolor{cvtitlecolor}{#1}}}
\newcommand{\cvdurationstyle}[1]{{\small\cvdurationfont\textcolor{cvdurationcolor}{#1}}}
\newcommand{\cvheadingstyle}[1]{{\normalsize\cvheadingfont\textcolor{cvheadingcolor}{#1}}}


% inter-item spacing
% ------------------

% vertical space after personal info and standard CV items
\newlength{\cvafteritemskipamount}
\setlength{\cvafteritemskipamount}{3mm plus 1.25mm minus 1.25mm}

% vertical space after sections
\newlength{\cvaftersectionskipamount}
\setlength{\cvaftersectionskipamount}{1.5mm plus 0.5mm minus 0.5mm}

% extra vertical space to be used when a section starts with an item with a heading (e.g. in the skills section),
% so that the heading does not follow the section name too closely
\newlength{\cvbetweensectionandheadingextraskipamount}
\setlength{\cvbetweensectionandheadingextraskipamount}{1mm plus 0.25mm minus 0.25mm}


% intra-item spacing
% ------------------

% vertical space after name
\newlength{\cvafternameskipamount}
\setlength{\cvafternameskipamount}{3mm plus 0.75mm minus 0.75mm}

% vertical space after personal info lines
\newlength{\cvafterpersonalinfolineskipamount}
\setlength{\cvafterpersonalinfolineskipamount}{2mm plus 0.5mm minus 0.5mm}

% vertical space after titles
\newlength{\cvaftertitleskipamount}
\setlength{\cvaftertitleskipamount}{1mm plus 0.25mm minus 0.25mm}

% value to be used as parskip in right column of CV items and itemsep in lists (same for both, for consistency)
\newlength{\cvparskip}
\setlength{\cvparskip}{0.5mm plus 0.125mm minus 0.125mm}

% set global list configuration (use parskip as itemsep, and no separation otherwise)
\setlist{parsep=0mm,topsep=0mm,partopsep=0mm,itemsep=\cvparskip}


% CV commands
% -----------

% creates a "personal info" CV item with the given left and right column contents, with appropriate vertical space after
% @param #1 left column content (should be the CV photo)
% @param #2 right column content (should be the name and personal info)
\newcommand{\cvpersonalinfo}[2]{
    % left and right column
    \begin{minipage}[t]{\cvleftcolumnwidth}
        \vspace{0mm} % XXX hack to align to top, see https://tex.stackexchange.com/a/11632
        \raggedleft #1
    \end{minipage}% XXX necessary comment to avoid unwanted space
    \hspace{\cvcolumngapwidth}% XXX necessary comment to avoid unwanted space
    \begin{minipage}[t]{\cvrightcolumnwidth}
        \vspace{0mm} % XXX hack to align to top, see https://tex.stackexchange.com/a/11632
        #2
    \end{minipage}

    % space after
    \vspace{\cvafteritemskipamount}
}

% typesets a name, with appropriate vertical space after
% @param #1 name text
\newcommand{\cvname}[1]{
    % name
    \cvnamestyle{#1}

    % space after
    \vspace{\cvafternameskipamount}
}

% typesets a line of personal info beginning with an icon, with appropriate vertical space after
% @param #1 parameters for the \includegraphics command used to include the icon
% @param #2 icon filename
% @param #3 line text
\newcommand{\cvpersonalinfolinewithicon}[3]{
    % icon, vertically aligned with text (see https://tex.stackexchange.com/a/129463)
    \raisebox{.5\fontcharht\font`E-.5\height}{\includegraphics[#1]{#2}}
    % text
    #3

    % space after
    \vspace{\cvafterpersonalinfolineskipamount}
}

% creates a "section" CV item with the given left column content, a horizontal rule in the right column, and with
% appropriate vertical space after
% @param #1 left column content (should be the section name)
\newcommand{\cvsection}[1]{
    % left and right column
    \begin{minipage}[t]{\cvleftcolumnwidth}
        \raggedleft\cvsectionstyle{#1}
    \end{minipage}% XXX necessary comment to avoid unwanted space
    \hspace{\cvcolumngapwidth}% XXX necessary comment to avoid unwanted space
    \begin{minipage}[t]{\cvrightcolumnwidth}
        \textcolor{cvrulecolor}{\rule{\cvrightcolumnwidth}{0.3mm}}
    \end{minipage}

    % space after
    \vspace{\cvaftersectionskipamount}
}

% creates a standard, multi-purpose CV item with the given left and right column contents, parskip set to cvparskip
% in the right column, and with appropriate vertical space after
% @param #1 left column content
% @param #2 right column content
\newcommand{\cvitem}[2]{
    % left and right column
    \begin{minipage}[t]{\cvleftcolumnwidth}
        \raggedleft #1
    \end{minipage}% XXX necessary comment to avoid unwanted space
    \hspace{\cvcolumngapwidth}% XXX necessary comment to avoid unwanted space
    \begin{minipage}[t]{\cvrightcolumnwidth}
        \setlength{\parskip}{\cvparskip} #2
    \end{minipage}

    % space after
    \vspace{\cvafteritemskipamount}
}

% typesets a title, with appropriate vertical space after
% @param #1 title text
\newcommand{\cvtitle}[1]{
    % title
    \cvtitlestyle{#1}

    % space after
    \vspace{\cvaftertitleskipamount}
    % XXX need to subtract cvparskip here, because it is automatically inserted after the title "paragraph"
    \vspace{-\cvparskip}
}


% header and footer
% -----------------

% set empty header and footer
\pagestyle{empty}



% preamble end/document start
% ===========================

\begin{document}


% personal info
% -------------

\cvpersonalinfo{
    % photo
    \includegraphics[height=38mm]{picture.JPG}
}{
    % name
    \cvname{Arianna Masciolini}

    % email address
    \cvpersonalinfolinewithicon{height=4mm}{070-envelop.pdf}{
        arianna.masciolini@gu.se;
        arianna@fripost.org
    }

    % address
    \cvpersonalinfolinewithicon{height=4mm}{072-location.pdf}{
        Tyghusvägen 8, lgh 1301, 41527 Gothenburg, Sweden
    }

    % phone number
    \cvpersonalinfolinewithicon{height=4mm}{067-phone.pdf}{
        +46 730358245;
        +39 3664750731
    }
    
    % GitLab
    %\cvpersonalinfolinewithicon{height=4mm}{gitlab.pdf}{
    %    gitlab.com/harisont
    %}
    % Website
    \cvpersonalinfolinewithicon{height=4mm}{website.pdf}{
        harisont.github.io
    }
    
    % GitHub
    \cvpersonalinfolinewithicon{height=4mm}{github.pdf}{
        github.com/harisont
    }

    % LinkedIn account
    %\cvpersonalinfolinewithicon{height=4mm}{458-linkedin.pdf}{
    %    arianna-masciolini-696a41139
    %}
}

% education
% ---------

\cvsection{EDUCATION}

% phd
\cvitem{
    \cvdurationstyle{February 2022 - present}%February 2021}
}{
    \cvtitle{PhD in Natural Language Processing}
    \\
    \textit{Språkbanken, Göteborgs Universitet (Gothenburg, Sweden)}
    \begin{itemize}[leftmargin=*]
        \item Main topic: Computational methods L2 grammar acquisition
    \end{itemize}
}

% master's
\cvitem{
    \cvdurationstyle{September 2018 - March 2021}%February 2021}
}{
    \cvtitle{Master's Degree in Computer Science}
    \\
    \textit{Göteborgs Universitet (Gothenburg, Sweden)}
    \begin{itemize}[leftmargin=*]
        \item Thesis: \textit{``Concept Alignment for Machine Translation''}
        \item Main subjects: Computational Linguistics, Functional Programming, Compilers
    \end{itemize}
}

% after master
\cvitem{
    \cvdurationstyle{September 2018 - December 2021}
}{
    \cvtitle{Freestanding University Courses}
    \\
    \textit{Göteborgs Universitet (Gothenburg, Sweden)}
    \begin{itemize}[leftmargin=*]
        \item Language-based security
        \item Parallel computer architecture
        \item Photography project
        \item Practical EU knowledge
    \end{itemize}
}

% bachelor's
\cvitem{
    \cvdurationstyle{October 2015 - July 2018}
}{
    \cvtitle{Bachelors's Degree in Computer Science}
    
    \textit{Università degli Studi di Perugia (Perugia, Italy)}

    \begin{itemize}[leftmargin=*]
        \item Thesis: \textit{``Algorithms for Energy Games: Parallel Implementation''}
    \end{itemize}
}

% publications

%\cvsection{PUBLICATIONS}
%\cvitem{
%    \cvdurationstyle{August 2021}
%}{
%    \cvtitle{Arianna Masciolini and Aarne Ranta, ``\textit{Grammar-Based Concept Alignment for Domain-Specific Machine Translation}''}
%    
%    \textit{Published in Proceedings of the Seventh International Workshop on Controlled Natural Language (CNL 2020/21)}
%}

% work experience
% ---------------

\cvsection{WORK EXPERIENCE}

\cvitem{
    \cvdurationstyle{November 2019 - present}
}{
    \cvtitle{Teaching}

    \textit{Göteborgs Universitet (Gothenburg, Sweden)} \
    \begin{itemize}[leftmargin=*]
    \item Programming courses: \textit{Introduction to Programming with Python}, {Advanced Programming with Python}, \textit{Introduction to Functional Programming} and \textit{Functional Programming},
    \item Computational Linguistics courses: Computational Syntax
    \item Project supervision: "Breaking Barriers: Enhancing Universal Dependency Parsing for Amharic" (Master's thesis in Language Technology)
    \end{itemize}
}

\cvitem{
    \cvdurationstyle{February 2022 - December 2022}
}{
    \cvtitle{Software Development}
    \textit{Göteborgs Universitet (Gothenburg, Sweden)}
    \begin{itemize}[leftmargin=*]
    \item Lärka language learning platform
    \end{itemize}
}

\cvitem{
    \cvdurationstyle{March 2021 - October 2021}
}{
    \cvtitle{Software Development}

    \textit{Digital Grammars AB (Gothenburg, Sweden)}
}

\cvitem{
    \cvdurationstyle{March 2017 - May 2018}
}{
    \cvtitle{Student Administration}

    \textit{Dipartimento di Matematica ed Informatica, Università degli Studi di Perugia (Perugia, Italy)}
    
}

%\cvitem{
%    \cvdurationstyle{Summer 2014}
%}{
%    \cvtitle{Bookbinding Apprentice}
%
%    \textit{Antica legatoria Biccini (Perugia, Italy)}
%}

% other experiences
% ------

\cvsection{OTHER EXPERIENCE}

\vspace{\cvbetweensectionandheadingextraskipamount}

\cvitem{
    \cvheadingstyle{Tutorials}
}{
    "Impara l'Haskell e mettilo da parte" (online Haskell videocourse, ongoing)
}

\cvitem{
    \cvheadingstyle{Private tutoring}
}{
    English, Latin, programming (Python and Kotlin)
}

\cvitem{
    \cvheadingstyle{Hackathons}
}{
    Code With Friends Fall \& Summer 2020, LibreOffice Perugia HackFest 2017, CSE Hackathon 2019
}

\cvitem{
    \cvheadingstyle{Translation}
}{
    Educational content on Khan Academy, interactive articles, popular science videos
}

% skills
% ------

\cvsection{SKILLS AND INTERESTS}

\vspace{\cvbetweensectionandheadingextraskipamount}

% languages
\cvitem{
    \cvheadingstyle{Languages}
}{
    Italian, English, Swedish, Spanish
}

\cvitem{
    \cvheadingstyle{Programming languages}
}{
    Python, Haskell, Grammatical Framework, Agda, Kotlin, Java, C
}

% job related skills
\cvitem{
    \cvheadingstyle{Other technologies}
}{
    GNU/Linux, Git, LaTeX
}

\cvitem{
    \cvheadingstyle{Personal interests}
}{
    Photography, literature, comics, early vocal music, bikepacking, skating, rock climbing 
}

\end{document}
